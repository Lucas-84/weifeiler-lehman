\documentclass{beamer}

\usepackage[frenchb]{babel}
\usepackage[T1]{fontenc}
\usepackage[utf8]{inputenc}
% \usepackage{latexsym}
% \usepackage{listings}

\usetheme{Warsaw}

\mode<presentation>

\title{Isomorphismes de graphes}
\author{Lucas Pesenti \& Martin Pépin}
\institute{Cours d'algorithmique}
\date{11 janvier 2017}

\AtBeginSection[] {
    \begin{frame}
        \frametitle{Sommaire}
        \tableofcontents[currentsection, hideothersubsections]
    \end{frame} 
}

\begin{document}

  \begin{frame}
    \titlepage
  \end{frame}
  
  \begin{frame}
    \frametitle{Sommaire}
    \tableofcontents[hideallsubsections]
  \end{frame}

\section{Isomorphisme de graphe}
  \subsection{Le problème}

    \begin{frame}
      \frametitle{Définition}
      \begin{block}{Isomorphisme}
        Deux graphes $G_1 = (V_1, E_1)$ et $G_2 = (V_2, E_2)$ sont isomorphes
        s'il existe une bijection $\rho : V_1 \to V_2$ qui préserve les arêtes
        \emph{e.g.} telle que $\rho(x)\rho(y) \in V_2$ si et seulement si
        $xy \in V_1$.
      \end{block}
      \pause
      Problème : on ne connait pas d'algorithme polynomial.
    \end{frame}

  \subsection{L'heuristique de Weisfeiler-Lehman}

    \begin{frame}
      \frametitle{Partition des sommets}
      Améliorations
      \begin{itemize}
        \item<1-> Première idée : partitionner les sommets selon leur degré
        \item<2-> Mieux : regarder les degrés des voisins
        \item<3-> Etc… $\to$ Partition stable
      \end{itemize}
      % TODO DESSIN
    \end{frame}

    \begin{frame}
      \frametitle{Weisfeiler-Lehman}
      \begin{block}{Heuristique de Weisfeiler-Lehman}
        Calculer la partition stable et appliquer l'algorithme glouton
        dans les partitions correspondantes.
      \end{block}
    \end{frame}

\section{Implémentation}

    \begin{frame}
      \frametitle{Structures des données}
      \begin{itemize}
        \item<1->
          Graphes
          \begin{itemize}
            \item listes d'adjacences
            \item On se souvient des degrés
          \end{itemize}
        \item<2->
          Partitions
          \begin{itemize}
            \item<2-> Liste d'adjacence
            \item<2-> Degrés
            \item<3-> Liste des nœuds de chaque part
            \item<3-> Map nœud $\to$ part
            \item<3-> Tailles des parts
            \item<3-> Nombre de part
          \end{itemize}
      \end{itemize}
    \end{frame}

    \begin{frame}
      \frametitle{Partitionnement simple}
    \end{frame}

    \begin{frame}
      \frametitle{Weisfeiler-Lehman}
    \end{frame}

\section{Résultats}

\end{document}
